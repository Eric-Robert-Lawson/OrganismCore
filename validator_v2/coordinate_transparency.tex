\documentclass[11pt]{article}
\usepackage{amsmath,amssymb,amsthm,hyperref,geometry,graphicx}
\geometry{a4paper, margin=1in}

\newtheorem{theorem}{Theorem}
\newtheorem{definition}{Definition}
\newtheorem{proposition}{Proposition}
\newtheorem{remark}{Remark}

\title{Coordinate Transparency and the Variable‑Count Barrier:\\
       A Computational Discovery on a Cyclotomic Hypersurface}
\author{Eric Robert Lawson\\
        OrganismCore Project\\
        OrganismCore@proton.me}
\date{\today}

\begin{document}
\maketitle

% ----------------------------------------------------------------------------
\section*{Abstract}
We report a computational discovery (verified modulo $p=313$; multi‑prime checks in progress) on a degree‑8 cyclotomic hypersurface $V \subset \mathbb{P}^5$:   cohomology classes in the 707‑dimensional Galois‑invariant primitive Hodge space can be represented in \emph{any} 4‑variable coordinate subring.  This phenomenon — which we call \emph{coordinate transparency} — invalidates a naive Newton‑polytope proof strategy for non‑algebraicity and requires the use of intrinsic algebraic invariants (cokernel membership, Galois‑trace rank) to prove non‑algebraicity rigorously. We describe the experiments, provide reproducible scripts, and outline the deterministic algebraic steps required to lift our modular evidence to characteristic zero.

% ----------------------------------------------------------------------------
\section*{Provenance \& Computational Scope}

\begin{quote}
All computations reported were performed \textbf{modulo $p = 313$} using Macaulay2 version 1.25.11 on macOS 12.6.   Scripts and raw outputs are archived at Zenodo DOI [INSERT‑DOI‑WHEN‑READY].  

\textbf{Scripts:  }  \texttt{validator\_v2/c1. m2}, \texttt{validator\_v2/c2.m2}.  

\textbf{Inputs: } 2590‑dimensional invariant monomial basis (Certificate C2, mod 313), 707‑dimensional cokernel, multiplication matrix.  

\textbf{Primes used: } $p=313$ (reported here); additional runs on $p \in \{53,79,131,157\}$ planned/in‑progress.  

\textbf{Runtime:} CP1 $\sim$15 min, CP2 $\sim$45 min.  

\textbf{Reproducibility:} see Section~\ref{sec:repro}. 
\end{quote}

% ----------------------------------------------------------------------------
\section{Introduction}

The Hodge conjecture asserts that on a smooth projective complex variety, certain rational cohomology classes (the Hodge classes) should be representable by algebraic cycles.   Proving or disproving this claim has remained one of the central open problems in algebraic geometry, and the Clay Mathematics Institute recognizes it as a Millennium Prize Problem. 

In earlier work [OrganismCore 2026] we identified 401 candidate Hodge classes on a degree‑8 cyclotomic hypersurface
\[
  V := \{ F = 0 \} \subset \mathbb{P}^5, 
  \quad F = \sum_{k=0}^{12} L_k^8, 
  \quad L_k = \sum_{j=0}^{5} \omega^{kj} z_j,
\]
where $\omega$ is a primitive 13th root of unity. These 401 classes exhibit strong empirical separation from the known 16 algebraic 2‑cycles (hyperplane and coordinate intersections):
\begin{itemize}
  \item \textbf{Variable‑count barrier: } All known cycles use $\le 4$ variables; candidates use all 6 (D=1. 000 separation).
  \item \textbf{Dimensional gap:} The Galois‑invariant primitive Hodge space $H^{2,2}_{\mathrm{prim,inv}}$ has dimension 707; known algebraic cycles span (conjecturally) $\le 12$ dimensions.
  \item \textbf{Complexity barrier:} Kolmogorov complexity of algebraic monomials is significantly lower (Cohen's $d=2. 22$, $p < 10^{-78}$).
\end{itemize}

The present note reports a computational experiment (Update 3) undertaken to test whether the variable‑count barrier could be converted into an algebraic impossibility theorem by proving that all algebraic classes in the Jacobian ring must have representatives using at most 4 variables (a \emph{Newton‑polytope obstruction}).

\textbf{What we found instead:} the opposite — coordinate transparency:   any Hodge class can be represented in \emph{any} 4‑variable subring modulo the Jacobian ideal.  This behavior is surprising, geometric, and implies that the correct route to rigorous non‑algebraicity must use intrinsic invariants (cokernel membership and Galois‑trace ranks) rather than the support of individual representatives.

% ----------------------------------------------------------------------------
\section{Computational Setup}

\subsection{Cyclotomic Hypersurface and Jacobian Ring}
Let $V \subset \mathbb{P}^5$ be as above. The primitive middle cohomology $H^{2,2}_{\mathrm{prim}}$ is isomorphic to a graded piece of the Jacobian ring (as shown by the Griffiths residue map). The Galois‑invariant subspace (under the $C_{13}$ action) lies in the cokernel of the multiplication matrix
\[
  M\colon R(F)_{11} \otimes J(F) \to R(F)_{18,\,\mathrm{inv}},
\]
which has dimension 707 over $\mathbb{Q}$ (Certificate C2; see prior work). Here $R(F)$ denotes the ring $\mathbb{Q}[z_0,\ldots,z_5] / (F)$, and $J(F)$ is the Jacobian ideal.

\subsection{Modular Reduction}
All experiments reported here were run \textbf{modulo $p=313$}, which is a good prime ($p \equiv 1 \pmod{13}$) that embeds the 13th roots of unity.   We use Macaulay2 to compute with the quotient ring over $\mathbb{F}_{313}$.  

% ----------------------------------------------------------------------------
\section{Experiments and Main Results}

\subsection{Checkpoint 1 (CP1): Naive Residue Reduction}

\textbf{Goal:} Test whether Jacobian reduction preserves variable support.

\textbf{Setup:} Take a degree‑18 monomial using only $\{z_2,z_3,z_4,z_5\}$ (i.e.   avoiding $z_0,z_1$) and reduce it modulo $J(F)$ using the Macaulay2 remainder operation. 

\textbf{Result:} The remainder polynomial included all six variables — the "forbidden" variables $z_0,z_1$ appeared in the reduced form. 

\textbf{Interpretation:} The canonical remainder (as computed by Macaulay2) is not a minimal‑support representative.  Variable count of a single polynomial is \emph{not} an invariant of the cohomology class.

\subsection{Checkpoint 2 (CP2): Ideal Membership Test}

\textbf{Goal:} Determine whether a candidate Hodge class can be represented by polynomials supported on \emph{any} subset of 4 variables.

\textbf{Method:} For the candidate 
\[
  m = z_0^9 z_1^2 z_2^2 z_3^2 z_4^1 z_5^2 \pmod{J(F)},
\]
we tested the ideal membership question
\[
  m \overset{?}{\in} (J(F) + I_{\mathrm{forbidden}}),
\]
where $I_{\mathrm{forbidden}}$ is the ideal generated by the 2 variables not in a given 4‑subset.

\textbf{Setup:} There are $\binom{6}{4} = 15$ such subsets. We ran Macaulay2's ideal membership test for each.  

\textbf{Result:  } \textbf{15/15 success} — the candidate is congruent modulo $J(F)$ to polynomials supported on \emph{every} 4‑variable subset. 

\textbf{Conclusion:} The class is \emph{coordinate transparent} mod 313: it has representatives in each 4‑var coordinate subring.  

\subsection{Theorem 1.1 (Computational)}

\begin{theorem}[Coordinate Transparency (Modular)]
Let $V$ be the cyclotomic hypersurface as above, and let $m = z_0^9 z_1^2 z_2^2 z_3^2 z_4 z_5^2$ be a degree‑18 monomial representing a Hodge class in the Jacobian ring $R(F)_{18} / J(F)$ over $\mathbb{F}_{313}$. 

Then, \textbf{modulo $p=313$}, the class $[m]$ is congruent to elements supported in \emph{every} subset of 4 variables (all 15 subsets).

That is, for every 4‑subset $S \subset \{z_0,\ldots,z_5\}$ there exists a polynomial $P_S$ using only the variables in $S$ such that
\[
  m \equiv P_S \pmod{J(F)}.
\]
\end{theorem}

\begin{remark}
The statements in Theorem 1.1 are computational assertions verified over the finite field $\mathbb{F}_{313}$. See Section~\ref{sec:repro} for scripts and outputs.  Lifting to characteristic zero requires multi‑prime CRT reconstruction and/or a theoretical proof; Section~\ref{sec:future} describes the deterministic path.  
\end{remark}

% ----------------------------------------------------------------------------
\section{Implications for the Hodge Conjecture Program}

\subsection{Invalidation of Polytope Route}

Our initial hypothesis (Update 2) was that algebraic cycles would be \emph{constrained} to low‑dimensional Newton polytopes, and that high‑dimensional monomials (using all 6 variables) would be provably non‑algebraic by a polytope obstruction. 

Theorem 1.1 shows this approach fails:   the ability to collapse to any 4‑subset means \emph{variable count of a single representative is not intrinsic}. 

\subsection{Correct Approach:   Intrinsic Invariants}

To rigorously prove a Hodge class is not algebraic, we must now:  
\begin{enumerate}
  \item \textbf{Compute cokernel coordinates} for the candidate and all known algebraic cycles (Griffiths residues) in the 707‑dim cokernel basis.
  \item \textbf{Test linear membership:  } Is candidate $\in \mathrm{span}_{\mathbb{Q}}(\text{16 known cycles})$?
  \item \textbf{Prove CH bound: } Show via Galois‑trace rank (or Shioda theory) that $\dim \mathrm{CH}^2(V)_{\mathbb{Q}} = 12$ (i.e.  16 cycles span exactly that 12‑dim space).
  \item \textbf{Conclude:  } If candidate $\not\in \mathrm{span}$ and $\dim(\mathrm{span}) = 12 = \dim(\mathrm{CH}^2)$, then candidate is non‑algebraic.
\end{enumerate}

This is a purely algebraic test and can be done deterministically via multi‑prime rank + CRT reconstruction. 

\subsection{Why Coordinate Transparency is Geometric}

The 15/15 success across all 4‑variable subsets (mod 313) is itself a \emph{strong geometric phenomenon}. It suggests a global symmetry/exchange property in the Jacobian ring induced by the $C_{13}$ action and the cyclotomic structure.   While we have not yet explained this theoretically, it likely follows from the representation theory of the symmetry group acting on the cohomology.

% ----------------------------------------------------------------------------
\section{Technical Detail:   The Ideal Membership Mechanism}

For completeness, we sketch how the membership test works.   For a given 4‑subset $S$ (e.g. $\{z_0,z_1,z_2,z_3\}$), let the "forbidden" variables be $F = \{z_4,z_5\}$.   We ask: 
\[
  m \overset{?}{\in} \bigl( J(F) + (z_4, z_5) \bigr).
\]
Macaulay2 computes the remainder
\[
  m \bmod \bigl(J(F) + (z_4,z_5)\bigr).
\]
If the remainder is zero, then $m$ is in the ideal — i.e. there exists a representative of $[m]$ that uses only variables in $S$ (all other variables vanish in the quotient by the "forbidden ideal").

We ran this for all 15 subsets and obtained remainder=0 for every one (mod 313).

% ----------------------------------------------------------------------------
\section{Related Work and Context}

Coordinate transparency is a rare phenomenon.   In generic varieties, classes are typically \emph{locked} to their support.  The flexible representational structure we observe is likely a consequence of the high‑symmetry cyclotomic construction and the fact that the Jacobian ideal generators (degree 7) are dense and involve all variables.  Prior work (e.g.  Shioda, Schoen) on Fermat and cyclotomic varieties provides partial insight but does not predict this exact behavior.

% ----------------------------------------------------------------------------
\section{Future Directions}
\label{sec:future}

\subsection{Multi‑Prime Verification and CRT Reconstruction}
We plan to run CP1/CP2 on at least two additional good primes (e.g. $p \in \{53,131\}$). If results agree modulo all primes, we will use the Chinese Remainder Theorem to reconstruct rational statements wherever possible. 

\subsection{Cokernel‑Membership Test}
We will compute the 16 algebraic cycles' coordinates in the 707‑dim cokernel basis and test whether each candidate is in the $\mathbb{Q}$‑span.   This is a deterministic linear algebra computation. 

\subsection{Galois‑Trace Rank Certificate}
We will compute the 16×16 Galois‑trace pairing matrix modulo several primes, reconstruct the integer matrix via CRT, and show $\mathrm{rank}(\text{pairing}) = 12$ over $\mathbb{Q}$. This gives the rigorous dimension bound for $\mathrm{CH}^2(V)$.

Combining these will yield a rigorous, unconditional non‑algebraicity proof for at least some candidates.

% ----------------------------------------------------------------------------
\section{Reproducibility}
\label{sec: repro}

All scripts and outputs are publicly archived at Zenodo [DOI to be inserted]. 

\subsection{Running the Scripts}
\textbf{Requirements:} Macaulay2 v1.25.11+, access to the saved Certificate C2 JSON outputs (2590‑dim monomial basis mod 313).

\textbf{Commands:}
\begin{verbatim}
m2 validator_v2/c1.m2 > c1_output_p313.txt
m2 validator_v2/c2.m2 > c2_output_p313.txt
\end{verbatim}

\textbf{Expected runtime:} CP1 $\sim$15 min, CP2 $\sim$45 min on modern laptop.

\subsection{Artifact List}
\begin{itemize}
  \item \texttt{c1.m2}, \texttt{c2.m2} (Macaulay2 scripts)
  \item \texttt{c1\_output\_p313.txt}, \texttt{c2\_output\_p313.txt} (full run logs)
  \item \texttt{c1\_output\_p313.json}, \texttt{c2\_output\_p313.json} (parsed data)
\end{itemize}

% ----------------------------------------------------------------------------
\section{Conclusion}

We have computationally observed \emph{coordinate transparency} on a degree‑8 cyclotomic hypersurface:   Hodge classes can be represented in any 4‑variable coordinate subring modulo the Jacobian ideal (verified mod $p=313$; multi‑prime checks pending). This invalidates a naive variable‑count obstruction approach but provides a deeper structural insight:   the correct route to rigorous non‑algebraicity is via cokernel membership and Galois‑trace rank.  These deterministic algebraic steps are now clearly defined and will be executed next. 

\subsection{Acknowledgments}
We thank Claude (Anthropic) and ChatGPT (OpenAI) for deep collaboration on logical debugging, certification design, and technical validation. This work is self‑funded and released under open‑access principles. 

% ----------------------------------------------------------------------------
\begin{thebibliography}{9}

\bibitem{shioda}
  T. ~Shioda,
  \emph{The Hodge conjecture for Fermat varieties},
  Math. Ann. \textbf{245} (1979), 175--184.

\bibitem{organismcore}
  E.~R. ~Lawson,
  \emph{Multi‑Barrier Evidence for Non‑Algebraic Hodge Classes on a Cyclotomic Hypersurface},
  Zenodo preprint (2026), DOI: 10.5281/zenodo.18292776.

\end{thebibliography}

\end{document}