\documentclass[11pt]{article}
\usepackage{amsmath, amssymb, amsthm}
\usepackage{geometry}
\usepackage{hyperref}
\geometry{margin=1in}

\newtheorem{definition}{Definition}[section]
\newtheorem{remark}{Remark}[section]

\title{Causal Equivalence and Compatibility:\\
A Subfield of the Universal Reasoning Substrate Theory (URST)}
\author{Eric Robert Lawson}
\date{\today}

\begin{document}

\maketitle

\begin{abstract}
This document introduces \textbf{Causal Equivalence and Compatibility} as a new subfield emerging from the Universal Reasoning Substrate Theory (URST). By defining reasoning spaces, derivative reasoning spaces, and RARFL as foundational constructs, we formalize a framework to measure and analyze causal structure across domains. Causal equivalence provides a continuous, measurable, and explainable substrate, while causal compatibility defines the structural preconditions necessary for equivalence to exist. This subfield enables precise cross-domain reasoning transfer, emergent optimization, and explainable causal analysis.
\end{abstract}

\section{Introduction}
The Universal Reasoning Substrate Theory (URST) provides a framework for modeling reasoning as a manipulable and objectifiable substrate. Within URST, reasoning spaces can be formalized, optimized, and transferred across domains via derivative reasoning spaces and the RARFL process.  

A natural corollary of URST is the concept of \textbf{causal equivalence}, which measures the degree to which reasoning trajectories across distinct domains produce equivalent causal outcomes. Closely related is \textbf{causal compatibility}, which defines the structural preconditions for equivalence to exist.

\section{Reasoning Spaces}
\begin{definition}[Reasoning Space]
A reasoning space $\mathcal{R}$ is a structured set of reasoning trajectories, where each trajectory represents a sequence of steps taken by an agent to achieve an objective under a specified reward function.
\end{definition}

\begin{definition}[Derivative Reasoning Space]
Given a reasoning space $\mathcal{R}$, a derivative reasoning space $\mathcal{R}'$ is a transformed space that encodes optimized or partially solved trajectories of $\mathcal{R}$, typically via RARFL or other reasoning optimization methods.
\end{definition}

\section{Causal Equivalence}
\begin{definition}[Causal Equivalence]
Two reasoning spaces $\mathcal{R}_1$ and $\mathcal{R}_2$ are \textbf{causally equivalent} over a subset of trajectories $S \subseteq \mathcal{R}_1 \times \mathcal{R}_2$ if the outcomes under corresponding trajectories produce equivalent effects relative to their respective reward functions.  

Formally, define a metric $\delta$ measuring divergence:
\[
\delta((\tau_1, \tau_2)) = \text{difference in causal effect of trajectories } \tau_1 \in \mathcal{R}_1, \tau_2 \in \mathcal{R}_2.
\]

Causal equivalence is satisfied when $\delta(\tau_1, \tau_2) \leq \epsilon$ for all $(\tau_1, \tau_2) \in S$, where $\epsilon$ is a threshold for acceptable divergence.
\end{definition}

\section{Causal Compatibility}
\begin{definition}[Causal Compatibility]
Two reasoning spaces $\mathcal{R}_1$ and $\mathcal{R}_2$ are \textbf{causally compatible} if there exists a non-empty subset of trajectories $S \subseteq \mathcal{R}_1 \times \mathcal{R}_2$ over which causal equivalence can be measured.  

If the reasoning spaces are incompatible, no mapping exists and causal equivalence is undefined.
\end{definition}

\begin{remark}
Causal compatibility defines the structural preconditions for equivalence. It distinguishes whether reasoning spaces are even comparable, while causal equivalence measures the degree of alignment where comparability exists.
\end{remark}

\section{Implications and Applications}
\begin{itemize}
    \item \textbf{Cross-domain transfer:} By quantifying equivalence and compatibility, reasoning optimized in one domain can be transferred to another with explicit measures of divergence.
    \item \textbf{Explainable AI:} Divergence regions provide interpretable explanations for where and why causal reasoning fails to transfer.
    \item \textbf{Emergent optimization:} RARFL and derivative reasoning spaces can leverage causal equivalence to guide optimization in partially observable or complex domains.
    \item \textbf{Foundational theory:} Introduces a continuous, measurable, and structured framework for reasoning-space causality, redefining the traditional binary causal inference.
\end{itemize}

\section{Conclusion}
Causal Equivalence and Compatibility, as a subfield of URST, provides a rigorous foundation for reasoning about causal relationships across domains. By shifting the perspective from binary inference to measurable equivalence, and introducing structural compatibility as a precondition, this subfield opens pathways for cross-domain reasoning transfer, emergent optimization, and explainable AI.  

This framework represents a corollary epiphany from URST, extending the theoretical reach of reasoning-space analysis and establishing a new, operationally measurable field for causality.

\end{document}
