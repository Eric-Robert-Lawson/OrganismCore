\documentclass[11pt]{article}
\usepackage{amsmath, amssymb, amsthm}
\usepackage{geometry}
\usepackage{hyperref}
\geometry{margin=1in}

\title{Subdomains of the Universal Reasoning Substrate Theory (URST):\\
Structure, Dynamics, and the Emergent Framework of Reasoning}
\author{Eric Robert Lawson}
\date{\today}

\begin{document}

\maketitle

\begin{abstract}
The Universal Reasoning Substrate Theory (URST) proposes that reasoning itself is a substrate with measurable structure, optimizable dynamics, and cross-domain compatibility constraints. This paper introduces and formalizes the emerging subdomains of URST, arising naturally from the interplay between reasoning spaces, derivative reasoning spaces, reward functions, and the RARFL process presented as a living document intended to evolve with communal exploration and theoretical development. These subdomains are not auxiliary considerations but fundamental degrees of freedom that make reasoning optimization, ethical alignment, causal measurement, and multi-agent coherence possible. This document consolidates and systematizes these subdomains, establishing a foundation for communal development, further research, and engineering of URST-based systems.
\end{abstract}

\section{A Living Document}

This article is a living document. Its purpose is to capture the current understanding of URST subdomains while remaining flexible to refinement. 
As the field develops, reasoning subdomains, mappings, and processes may be clarified, extended, or reorganized. 
This framework is intended to evolve with communal exploration, experimentation, and theoretical development, rather than remain a static exposition.


\section{Introduction}

The Universal Reasoning Substrate Theory (URST) describes reasoning as a universal structure that can be objectified, optimized, transferred, and measured. Its core consists of:

\begin{itemize}
    \item Reasoning spaces $\mathcal{R}$: the set of all valid reasoning trajectories.
    \item Derivative reasoning spaces $\mathcal{R}'$: optimized or partially solved trajectories.
    \item Reward functions $J$: incentive structures that guide optimization.
    \item The RARFL process: a triadic update cycle mapping $(\mathcal{R}, J)$ to $\mathcal{R}'$ and back.
\end{itemize}

From these foundations emerge entire subdomains of reasoning science---structures of inquiry that are not arbitrary but necessary consequences of URST itself. The present paper formalizes these subdomains and clarifies their boundaries. These subfields establish the groundwork for communal development, future mathematical formalization, and applied implementations within the OrganismCore project.

\section{The Reasoning Optimization Triad}

A core structural insight in URST is that reasoning optimization occurs through a triad of interacting components:

\begin{enumerate}
    \item \textbf{Objectification}: Encoding reasoning as composable structures (layering, POT functions, trajectories).
    \item \textbf{Operationalization}: Navigation primitives for traversing or constructing reasoning paths.
    \item \textbf{Contextualization}: Integration of semantic grounding, domain constraints, or world models.
\end{enumerate}

A second instantiation of the triad appears within RARFL:

\[
(\mathcal{R}, J) \longrightarrow \mathcal{R}' \longrightarrow \text{updated } J \longrightarrow \text{updated } \mathcal{R}
\]

RARFL is therefore a higher-order optimization operator defined over reasoning itself.

\subsection{Fixed-Point Structure}

Although this triad contains circular dependencies, they are not pathological. Instead, they correspond to \emph{fixed-point phenomena}:

\begin{itemize}
    \item Attractors in reasoning-space optimization.
    \item Stable equilibria of reward/trajectory interaction.
    \item Basins of convergence defining valid reasoning identities.
\end{itemize}

Thus, the triadic structure is generative, not contradictory.

\section{Causal Equivalence and Causal Compatibility}

URST implies a new formalism for causality:

\begin{itemize}
    \item \textbf{Causal Equivalence}: Two reasoning spaces are equivalent when the optimal trajectories under their respective rewards yield aligned derivative reasoning spaces.
    \item \textbf{Causal Compatibility}: The structural condition for equivalence to be definable at all.
\end{itemize}

This reframes causal inference as a measurement of deviation from equivalence.

\[
\text{Causal Divergence} = \delta(\mathcal{R}_1', \mathcal{R}_2')
\]

Causal inference becomes quantifiable, auditable, and explainable by locating mismatches between reasoning spaces and their derivative structures.

\section{Reward Function Fitting}

Ethics, justice, fairness, and all normative systems correspond to \emph{reward function fitting} in URST.

Given two elements of the triad, the third is derivable:

\begin{enumerate}
    \item $(\mathcal{R}, \mathcal{R}') \Rightarrow J$ (reward inference)
    \item $(J, \mathcal{R}) \Rightarrow \mathcal{R}'$ (RARFL optimization)
    \item $(J, \mathcal{R}') \Rightarrow \mathcal{R}$ (inverse reasoning space construction)
\end{enumerate}

This creates a formal substrate for normative reasoning. Ethical questions become optimization problems over:

\[
J^* = \arg\min_{J} \delta(\mathcal{R}'_J, \mathcal{R}'_{\text{intended}})
\]

where the divergence metric measures deviation from intended values.

\section{Optimization of RARFL Itself}

RARFL is not merely a process applied to reasoning spaces; it is itself a reasoning process subject to optimization.

With sufficient solved reasoning spaces, one can:

\begin{itemize}
    \item Build meta-derivative spaces describing how reward fitting evolves.
    \item Infer meta-reward structures governing good incentive design.
    \item Discover regularities in successful reasoning-space convergence.
\end{itemize}

This gives rise to:

\begin{center}
\textbf{Meta-RARFL}: Optimization of the optimization operator.
\end{center}

This corresponds to what is informally understood as strategic intelligence, wisdom, foresight, and meta-reasoning.

\section{Reasonable vs. Unreasonable}

URST implies a clean definitional boundary:

\begin{itemize}
    \item If a process can be represented within URST, it is reasonable.
    \item If a process cannot be represented within URST, it is not reasoning.
\end{itemize}

Thus, ``unreasonable'' behaviors are precisely those that fail to admit:

\[
(\mathcal{R}, J, \mathcal{R}') \text{ structure}
\]

This forms a universal criterion for reasoning-admissibility across domains.

\section{Multi-Agent Systems and Adversarial Dynamics}

Compatibility extends beyond single-agent reasoning spaces. For multi-agent systems:

\[
\mathcal{R} = \prod_i \mathcal{R}_i, \quad 
J = \sum_i J_i^{(\text{local})} + J_{\text{global}}
\]

RARFL constrains:

\begin{itemize}
    \item adversarial incentives,
    \item cooperative equilibria,
    \item market-like structures,
    \item social dynamics,
\end{itemize}

and gives formal tools for diagnosing misalignment or instability.

\section{Science as RARFL}

Scientific epistemology emerges naturally from URST:

\begin{itemize}
    \item Reasoning space = theories, models, hypotheses.
    \item Reward function = predictive accuracy, coherence, simplicity.
    \item Derivative reasoning space = validated scientific knowledge.
\end{itemize}

Thus, science itself is a RARFL cycle constrained by empirical reward feedback. URST formalizes how scientific reasoning converges.

\section{Toward a Complete Unified Reasoning Framework}

As URST matures, the space of:
\begin{itemize}
    \item allowable reasoning spaces,
    \item allowable reward functions,
    \item allowable derivative structures
\end{itemize}
begins to stabilize.

This suggests the possibility that the universe of valid reasoning is finite or convergent, with diminishing returns beyond certain foundational optimizations---a form of \emph{reasoning determinism}.

\section{Conclusion}

The subdomains presented here illustrate the inherent mathematical richness of URST. They are not optional modules but emergent necessities. As the OrganismCore project grows, these subdomains will guide communal research, formalization, and engineering efforts. The framework laid out here establishes a foundation for future collaboration aimed at mapping, refining, and extending the universal reasoning substrate.

Readers are invited to contribute insights, extensions, and refinements to these subdomains, helping to collectively advance the URST framework.

\section*{Glossary of Terms}

\begin{tabular}{|p{4cm}|p{10cm}|}
\hline
\textbf{Term} & \textbf{Definition} \\
\hline
Reasoning Space ($\mathcal{R}$) & The set of all valid reasoning trajectories within a domain or problem space. \\
\hline
Derivative Reasoning Space ($\mathcal{R}'$) & Optimized or partially solved reasoning trajectories derived from $\mathcal{R}$ via the RARFL process. \\
\hline
Reward Function ($J$) & Incentive structures that guide reasoning optimization and evaluation. \\
\hline
RARFL & Triadic process mapping $(\mathcal{R}, J) \to \mathcal{R}' \to J \to \mathcal{R}$, enabling iterative reasoning optimization. \\
\hline
Meta-RARFL & Optimization applied to the RARFL process itself, producing higher-order reasoning strategies. \\
\hline
Causal Equivalence & When two reasoning spaces with respective reward functions yield aligned derivative reasoning spaces. \\
\hline
Causal Divergence & Quantitative measure of deviation between derivative reasoning spaces, indicating causal non-equivalence. \\
\hline
\end{tabular}

\vspace{1em}

\noindent\textbf{Versioning Note:} This document is a living resource. Updates and refinements to URST subdomains are tracked via the associated \href{https://github.com/YourGitHubRepo}{GitHub repository}, where contributors can view the full history of changes, propose modifications, and discuss revisions. Users are encouraged to reference specific commits when citing subdomain definitions or processes.


\end{document}
