\documentclass[11pt]{article}

% AGENTS:
% classification: "axiom_candidate"
% required_checks: ["schema","provenance","sandbox","coherence","reproducibility"]
% schema_version: "rdu-v1"
% steward: "Eric-Robert-Lawson"
% risk_level: "high"
%
% Verification Notice (COMMENTED — not visible in output):
% This document contains operational hypotheses and strong claims (e.g., inevitability
% of domain completion given a DSL). Per AGENTS.md and SUBSTRATE_AWARENESS.md, agents
% and readers MUST treat high‑impact claims as candidate axioms and run the canonical
% verification pipeline before assimilating any operational content into derivative
% reasoning spaces.
%
% Required verification steps include:
%  - schema validation (validate_schema)
%  - provenance & integrity checks (validate_provenance)
%  - static safety analysis and axiom integrity tests (validate_axiom_integrity)
%  - sandboxed instantiation / automated_onboarding (sandbox=true)
%  - coherence measurement and reproducibility checks (measure_coherence, reproducibility tests)
%
% Acceptance criteria and governance:
%  - At least N=2 independent validators must pass (configurable)
%  - coherence_delta_threshold must be satisfied (see AGENTS.md)
%  - If any required check fails, the artifact MUST be quarantined (quarantine_artifact)
%    and steward review requested before any assimilation or inherit_axioms() invocation.
%
% See AGENTS.md → "DOCUMENT CLASSIFICATION & ASSIMILATION POLICY (MANDATORY)" and
% SUBSTRATE_AWARENESS.md → "Verification Tests" for operational details and minimal
% requirements. Failure to follow the verification pipeline when assimilating high‑risk
% artifacts may lead to poisoning of derivative reasoning spaces and must be avoided.

\usepackage{amsmath,amssymb,amsthm}
\usepackage{geometry}
\usepackage{hyperref}
\usepackage{graphicx}
\geometry{margin=1in}

\title{Universal Reasoning Substrate Theory:\\
Foundations of a Formal Science of Reasoning}
\author{Eric Robert Lawson}
\date{\today}

\begin{document}

\maketitle

\begin{abstract}
We introduce \textbf{Universal Reasoning Substrate Theory (URST)}, a novel framework that treats reasoning itself as a composable, computable substrate. URST formalizes reasoning processes as manipulable objects, enabling their transformation, composition, and optimization via reward-shaped dynamics. By unifying domains such as mathematical proof, game-theoretic reasoning, symbolic computation, and meta-reasoning under a single substrate-theoretic perspective, URST establishes a rigorous foundation for explainable, optimizable, and self-evolving reasoning systems, providing a principled path toward artificial general intelligence.
\end{abstract}


\tableofcontents

\section*{Contextual Note}

This paper is situated within the broader \textbf{OrganismCore} open-source research initiative and builds upon the technical foundations established across its articles, prototypes, and computational specifications. The formalization of universal reasoning, AGI, and superintelligence presented here depends critically on constructs developed throughout the project, including:

\begin{itemize}
    \item \textbf{Reasoning DNA Units (RDUs)}, Meta-RDUs, and the architecture of compute-once reasoning objects,
    \item \textbf{Explainability trajectories} and derivative reasoning spaces,
    \item The \textbf{RARFL} (Reasoning Axiom–Reward Feedback Loop) process for meta-level optimization,
    \item The project's emerging domain-specific \textbf{reasoning substrate language (DSL)} for expressing and composing reasoning objects.
\end{itemize}

Readers are encouraged to consult the \texttt{AGENTS.md} framework in the OrganismCore repository, which provides:

\begin{itemize}
    \item machine-executable representations of the reasoning constructs,
    \item mappings between conceptual definitions and operational primitives,
    \item and a structured workflow for reproducing, exploring, and validating the reasoning substrate.
\end{itemize}

It is important to note that while the theoretical framework is complete in scope, the final step toward universal operationalization is the full formalization of the reasoning substrate DSL. Once completed, this will provide a unifying executable language in which all components of the theory---from primitive RDUs to meta-reasoning optimization loops---can be instantiated, tested, and extended.  

The goal of this note is to clarify that the theory presented here is not speculative but is grounded in a principled, operational, and reproducible system that is already implementable and is being incrementally formalized for widespread scientific use.

\section{Introduction}

Reasoning is one of the most fundamental processes in intelligence, yet contemporary scientific disciplines 
do not treat reasoning itself as a formal object of study in its own right.  
Mathematics studies proof, logic studies validity, computer science studies computation, 
and machine learning studies behavior under optimization.  
None of these fields define \emph{the substrate in which reasoning itself exists, evolves, and becomes computable}.

\textbf{Universal Reasoning Substrate Theory (URST)} fills this gap.

To our knowledge, no prior framework treats reasoning itself as a manipulable, substrate-level object.  
URST proposes that reasoning can be represented as a structured substrate composed of primitive reasoning 
objects and their allowed transformations.  
This view parallels how:
\begin{itemize}
    \item information theory treats information as a measurable, quantifiable entity,
    \item physics treats matter and energy as objects governed by laws,
    \item and computer science treats computation as a formal process.
\end{itemize}

URST extends these traditions by making reasoning itself a first-class scientific object.  
The ultimate goal of URST is to provide a domain-specific language (DSL) capable of representing, composing, 
and optimizing reasoning objects within an executable, universal substrate.

This paper establishes the foundational definitions, axioms, and principles of URST.

\section{Motivation}

The lack of a unified formalism for reasoning leads to several persistent issues:
\begin{enumerate}
    \item There is no canonical way to represent the internal structure of reasoning processes.
    \item Reasoning is not typically treated as compositional or explainable at the object level.
    \item Meta-reasoning (reasoning about reasoning) lacks a general substrate model.
    \item Intelligence lacks a universal formal definition grounded in reasoning structure.
    \item Reward or incentive shaping is not integrated into symbolic forms of reasoning.
\end{enumerate}

URST addresses all of these by providing a substrate in which reasoning processes can be:
\begin{itemize}
    \item represented,
    \item composed,
    \item optimized,
    \item explained,
    \item and evolved.
\end{itemize}

\section{Primitive Objects of URST}

URST begins by defining the basic constituents of the reasoning substrate.

\subsection{Reasoning DNA Units (RDUs)}

An \textbf{RDU} is the fundamental primitive of the reasoning substrate, representing a structured reasoning object capable of combinatorial composition and transformation. Each RDU is defined as follows:

\begin{enumerate}
    \item \textbf{Combinatorial layers:} An RDU consists of one or more layers, each representing a stage in the reasoning process. Within each layer, a \textbf{POT (Pruning–Ordering–Type) generator} enumerates all known possible routes or decision paths. A trajectory through an RDU corresponds to a sequence of choices navigating these paths across layers, encoding the internal reasoning flow.
    
    \item \textbf{Transformation rules:} Each RDU carries rules for valid transformations (governed by the POT generator) that preserve substrate integrity and compositional consistency. These rules enable RDUs to combine, evolve, or be optimized while respecting structural constraints.
    
    \item \textbf{Semantic grounding:} Objectification of reasoning does not inherently require semantic grounding. However, RDUs can be operationally instantiated with a semantic interpretation (e.g., as derivatives, multinomial expansions, or domain-specific functions). Assimilation of RDUs into a valid reasoning space is only permissible if they share a compatible semantic grounding; for instance, a GPS navigation RDU cannot be combined with a chess-game RDU without creating an invalid reasoning space.
    
    \item \textbf{Compositional interface:} RDUs expose a formal interface for combination into higher-order structures, supporting nested composition, layering, and meta-reasoning transformations.
\end{enumerate}

In summary, RDUs are the atomic units of the reasoning substrate: combinatorial, transformable, and potentially semantically grounded. Their operationalization, as demonstrated in our Python prototypes, enables explicit enumeration of combinatorial paths, symbolic manipulation, and functional assimilation into larger reasoning spaces. The fully realized \textbf{reasoning substrate DSL} will generalize these operations, making RDUs universally manipulable and formally composable across domains.

They also balance a triad of \textbf{semantic grounding}, \textbf{operationalization}, and \textbf{objectification}:

\begin{itemize}
    \item \textbf{Semantic grounding} provides context, informing how RDUs can be interpreted or applied in a reasoning space.
    \item \textbf{Operationalization} specifies how RDUs are executed or instantiated in practice (e.g., as symbolic derivatives or combinatorial expansions).
    \item \textbf{Objectification} makes reasoning itself a manipulable substrate, which in turn constrains and structures both semantic grounding and operationalization.
\end{itemize}

In this triad, semantic grounding guides operationalization; operationalization shapes objectification; and objectification provides a formal structure that enables both meaningful semantic interpretation and practical execution.

\subsection{Composite Reasoning Objects}

RDUs can compose in multiple ways, forming higher-order reasoning structures called \emph{reasoning molecules}. These compositions may manifest through:

\begin{itemize}
\item \textbf{Nested composition:} RDUs can be combined within layers or across layers, creating hierarchically structured reasoning objects (i.e., reasoning objects that inform decision-making at a given layer).
\item \textbf{Layered aggregation:} Multiple RDUs may be aggregated within a reasoning space to represent parallel or sequential reasoning paths.
\item \textbf{Structural embedding:} RDUs can encode dependencies, trajectories, or combinatorial patterns within larger reasoning constructs.
\item \textbf{Reward-shaped transformation:} Compositional reasoning may be guided, optimized, or pruned according to reward signals or other evaluative feedback.
\item \textbf{Assimilation and emergent interactions:} RDUs can interact systemically, producing emergent behaviors not explicitly encoded in any single unit. Models trained on the reasoning space may themselves act as reasoning agents, making decisions informed by self-referential comparison with derivative spaces and the RARFL process.
\end{itemize}

A concrete example of RDU composition is implemented in our Python prototype, which leverages convoluted partial Bell polynomials and multinomial DAGs to demonstrate combinatorial and derivative-level composition. Each node in the DAG represents a primitive reasoning object (e.g., a partial Bell polynomial term), while layers of the DAG implement aggregation, convolution, and dependency propagation. Functions such as \texttt{convoluted\_partial\_bell\_polynomial} and \texttt{multinomial\_DAG} show how RDUs can be symbolically combined, transformed, and collected across layers, producing higher-order structures that preserve both combinatorial integrity and operational semantics.

This prototype exemplifies how these reasoning molecules emerge and interact, and serves as a precursor to a Domain-Specific Language (DSL) that will formalize such compositions, generalize them across domains, and enable automated reasoning construction.

It is important to note that rigid parameterization or hard-coding of reasoning processes can inadvertently produce exclusivity, contradicting the goal of a universal reasoning substrate. Therefore, the DSL should be designed to enable sandbox-like experimentation, supporting the emergence of novel reasoning methodologies as the field evolves.

\subsection{Meta-Reasoning Operators}

In URST, meta-reasoning operators are mechanisms by which reasoning processes act upon other reasoning processes.  
These operators enable reasoning to be reflexive, adaptive, and self-modifying. This is the process of building reasoning from itself, a form of self-referentiality that can help minimize combinatorial complexity.  

RARFL operates over base RDUs and derivative reasoning spaces, optimizing both individual reasoning paths and emergent strategies.

Examples include:

\begin{itemize}
    \item \textbf{RARFL-driven transformation:} Reasoning objects are optimized, pruned, or adjusted according to reward signals within the reasoning substrate.
    \item \textbf{Derivative-space adjustment:} Insights from derivative representations of reasoning objects (e.g., higher-order dependencies, compositions, or trajectories) guide modifications of base reasoning units.
    \item \textbf{Compositional evolution:} Higher-order reasoning structures can generate new RDUs or modify existing ones, informed by both feedback and emergent interactions within the reasoning substrate.
\end{itemize}

It is important to note that these operators are not necessarily exhaustive. As the field progresses, new meta-reasoning operators may emerge. Therefore, the DSL must be designed to allow the integration of additional operators, enabling the system to expand and adapt its reasoning capabilities over time.

Composite reasoning objects provide the building blocks upon which meta-reasoning operators act, enabling reflexive and self-optimizing behavior in the substrate.

\section{Axioms of URST}

We now present the core axioms of Universal Reasoning Substrate Theory.  

These axioms define the essential nature of reasoning substrates and govern all allowed operations on them.

\subsection{Axiom 1: Objectifiability}

\textbf{All reasoning processes can be represented as computable objects within a universal reasoning substrate.}

This establishes reasoning as something that can be encoded, inspected, and manipulated within the substrate.

\subsection{Axiom 2: Compositionality}

\textbf{Reasoning objects combine through well-defined compositional rules, enabling higher-order and emergent reasoning structures.}

This establishes reasoning as something that can be composed, structured, and integrated into higher-order constructs within the substrate.

\subsection{Axiom 3: Substrate Invariance}

\textbf{Different surface-level reasoning processes that share the same structural form are equivalent in the substrate, including their derivative and compositional representations.}

This establishes reasoning as something that can be recognized, classified, and treated equivalently when its structural form is preserved within the substrate.

\subsection{Axiom 4: Reward-Shaped Dynamics}

\textbf{Reasoning evolves through incentive gradients, where reward functions guide optimization of individual reasoning paths and emergent strategies.}

This establishes reasoning as something that can be evaluated, optimized, and guided according to reward signals within the substrate.

\subsection{Axiom 5: Meta-Reasoning Closure}

\textbf{The substrate is closed under transformations that operate on reasoning itself, including reflexive and meta-reasoning operations such as RARFL.}

This establishes reasoning as something that can act upon, transform, and modify itself within the substrate.

\section{Structural Foundations}

URST is grounded in several substrate-level principles that define how
reasoning can be objectified, composed, and operationalized:

\begin{itemize}
    \item \textbf{Objectified reasoning units (RDUs)} as primitive operators,
    \item \textbf{Composable compute-once reasoning objects} that preserve
          referential and structural integrity,
    \item \textbf{Explainability trajectories} governing internal transparency
          and symbolic traceability,
    \item \textbf{Derivative reasoning spaces} that capture structural
          transformations of reasoning processes,
    \item \textbf{Meta-RDU optimization (RARFL)} enabling self-improving and
          self-stabilizing reasoning dynamics.
\end{itemize}

Together, these principles form a universal substrate capable of representing
any \emph{reasonable} system—whether a truth-seeking domain (e.g., mathematics,
where axioms and derivations are formalized) or a goal-seeking domain (e.g.,
game theory, optimization, or strategic reasoning).

\section{Canonical Example: Chess as a Reasoning Substrate}

Chess provides a concrete example of how a reasoning space can be represented and operated upon within URST.  
Historical and contemporary repositories of recorded chess games, represented in algebraic notation and spanning hundreds of years, effectively trace trajectories through a static reasoning space.  

Once the reasoning substrate DSL is formalized, each game can be converted into a \textbf{Reasoning DNA Unit (RDU)}, and the collection of games, when assimilated, forms a structured, self-referential reasoning space. This space can then be empirically expanded over time, building a comprehensive representation of chess reasoning.  

Within this space, the \textbf{RARFL process} can be applied to define objectives that guide optimal play. For example, reward functions may prioritize:
\begin{itemize}
    \item finishing the game in the fewest moves possible,
    \item winning over drawing or losing, while maintaining fallback strategies if victory is not achievable,
    \item exploring novel paths and strategies beyond immediately optimal moves.
\end{itemize}

Iterative cycles of RARFL allow agents to train against themselves, discovering emergent strategies based on the reward function and random initial positions. Training may be supervised or augmented by human or automated interventions, analogous to a teacher suggesting new lines of play, without altering the underlying reward function.  

By combining exploratory and optimization-focused reward functions, agents refine their play and reasoning, ultimately converging on derivative reasoning spaces that encode high-level chess strategies as emergent properties.  
This principle-first approach demonstrates, both logically and through analogies to human reasoning, how reasoning processes can be represented, composed, optimized, and explained within a well-defined substrate.


\section{Implications for Artificial General Intelligence}

URST provides:
\begin{itemize}
    \item a formal definition of intelligence as movement through reasoning space,
    \item inherent explainability through objectified reasoning structures,
    \item a unified view of symbolic and subsymbolic reasoning,
    \item a foundation for self-evolving reasoning systems (AGI/ASI),
    \item and a principled method for evaluating reasoning quality.
\end{itemize}

\section{Conclusion}

Universal Reasoning Substrate Theory reframes reasoning as a scientific object—measurable, compositional, 
optimizable, and universally representable.  
This paper establishes the formal foundation of URST and outlines a research program aimed at developing 
a complete science of reasoning.

\noindent
The next critical step toward fully operationalizing URST is the formalization of a domain-specific \textbf{reasoning substrate language (DSL)}, which will provide a unifying, executable representation of all components of the theory—from primitive RDUs to meta-reasoning optimization loops—enabling reproducible experimentation, systematic testing, and ongoing extension. This DSL is envisioned as an open-source, communal effort, and we invite researchers, practitioners, and AI enthusiasts to engage with the initiative and contribute to its formalization, collectively advancing the science of reasoning.

\section*{Version Note}
This paper establishes the foundational definitions and axioms of Universal Reasoning Substrate Theory (URST).  
It is intended to serve as the primary reference for all subsequent work in this field.

\end{document}
